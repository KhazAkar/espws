\documentclass[12pt,a4paper]{scrartcl}
\usepackage{polski}
\usepackage[top=2.5cm, bottom=2.5cm, left=3.5cm, right=1cm]{geometry}
\usepackage{amsmath}
\usepackage{amsfonts}
\usepackage{amssymb}
\usepackage{graphicx}
\usepackage{titling}
\usepackage{titlesec}
\usepackage{fontspec}
\usepackage{siunitx}
\sisetup{math-micro=\text{µ},text-micro=µ}
\newfontfamily{\tytulyrozdzialow}{Arial}
\setmainfont{Times New Roman}
\begin{document}
	\begin{titlepage}
		\centering
		\includegraphics[scale=0.1]{images/logo.jpg}\par\vspace{1cm}
		{\scshape\LARGE Kierunek: Elektronika i Telekomunikacja\\ Specjalność: Teleinformatyka \par}
		\vspace{1cm}
		{\scshape\Large Praca dyplomowa inżynierska\par}
		\vspace{1.5cm}
		{\huge\bfseries Stacja meteorologiczna oparta o ESP8266\par}
		\vspace{2cm}
		{\Large\itshape Damian Zaręba\\Nr albumu 8389\par}
		\vfill
		Promotor:\par
		dr inż. Tadeusz Leszczyński
		
		\vfill
		
		% Bottom of the page
		{\large Mława 2019r. \par}
	\end{titlepage}
\tableofcontents
\newpage
\begin{section}	{\fontsize{14pt}{16.8pt} \tytulyrozdzialow Wstęp}
	\hspace{\parindent} Założeniem pracy jest stworzenie stacji meteorologicznej opartej o mikroprocesor ESP8266, złożonej z kilku modułów. Tymi elementami są:
	\begin{itemize}{}{}
		\item Płyta główna z mikrokontrolerem ESP8266EX dla przetwarzania informacji z sensorów oraz mikrokontrolerem ATtiny44 dla sterowania zasilaniem całego urządzenia;
		\item Samodzielnie wykonany anemometr ultradźwiękowy do pomiaru kierunku i prędkości wiatru;
		\item Sensor firmy BOSCH o nazwie BME280, który służy do odczytu temperatury, ciśnienia i wilgotności powietrza;
		\item Sensor firmy PLANTOWER o nazwie PMS7003, który mierzy ilość pyłu zawieszonego w powietrzu, o wielkości PM1.0, PM2.5 oraz PM10, mierzone w \si\micro g/m$^{3}$.
	\end{itemize}
	\par Pierwszą częścią pracy jest schemat blokowy urządzenia oraz ogólny opis poszczególnych modułów wykorzystanych do zbudowania tego urządzenia, wliczając w to charakterystyki głównych komponentów dla każdego modułu. Udokumentowane zostanie również skonfigurowanie środowisk, które zostały wykorzystane do stworzenia tego projektu.
	\par Następnie przejdę do analizy schematu urządzenia, a konkretnie płyty głównej, bazy z mikroprocesorami i zasilaniem dla wykorzystanych sensorów. Poddana dokładnej analizie będzie każda z części schematu, takie jak sekcja zasilania czy połączeniowa między płytą główną a sensorami.
	\par W kolejnym etapie pracy przedstawię kody źródłowe do wykorzystanych mikrokontrolerów i dokładnie je omówię, wraz z algorytmami wykorzystanymi do ich napisania.
	\par Następnie przedstawię w skrócie projekt anemometru ultradźwiękowego, do pomiaru prędkości i kierunku wiatru. Omówiony zostanie schemat elektryczny i blokowy urządzenia.
	\par Ostatnią częścią projektu będzie ukazanie działania stacji na przykładzie zdjęć urządzenia i zrzutów ekranu z interfejsu do jego obsługi.
\end{section}

\end{document}